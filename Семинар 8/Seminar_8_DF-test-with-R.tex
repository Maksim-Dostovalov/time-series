% Options for packages loaded elsewhere
\PassOptionsToPackage{unicode}{hyperref}
\PassOptionsToPackage{hyphens}{url}
%
\documentclass[
]{article}
\usepackage{amsmath,amssymb}
\usepackage{iftex}
\ifPDFTeX
  \usepackage[T1]{fontenc}
  \usepackage[utf8]{inputenc}
  \usepackage{textcomp} % provide euro and other symbols
\else % if luatex or xetex
  \usepackage{unicode-math} % this also loads fontspec
  \defaultfontfeatures{Scale=MatchLowercase}
  \defaultfontfeatures[\rmfamily]{Ligatures=TeX,Scale=1}
\fi
\usepackage{lmodern}
\ifPDFTeX\else
  % xetex/luatex font selection
\fi
% Use upquote if available, for straight quotes in verbatim environments
\IfFileExists{upquote.sty}{\usepackage{upquote}}{}
\IfFileExists{microtype.sty}{% use microtype if available
  \usepackage[]{microtype}
  \UseMicrotypeSet[protrusion]{basicmath} % disable protrusion for tt fonts
}{}
\makeatletter
\@ifundefined{KOMAClassName}{% if non-KOMA class
  \IfFileExists{parskip.sty}{%
    \usepackage{parskip}
  }{% else
    \setlength{\parindent}{0pt}
    \setlength{\parskip}{6pt plus 2pt minus 1pt}}
}{% if KOMA class
  \KOMAoptions{parskip=half}}
\makeatother
\usepackage{xcolor}
\usepackage[margin=1in]{geometry}
\usepackage{color}
\usepackage{fancyvrb}
\newcommand{\VerbBar}{|}
\newcommand{\VERB}{\Verb[commandchars=\\\{\}]}
\DefineVerbatimEnvironment{Highlighting}{Verbatim}{commandchars=\\\{\}}
% Add ',fontsize=\small' for more characters per line
\usepackage{framed}
\definecolor{shadecolor}{RGB}{248,248,248}
\newenvironment{Shaded}{\begin{snugshade}}{\end{snugshade}}
\newcommand{\AlertTok}[1]{\textcolor[rgb]{0.94,0.16,0.16}{#1}}
\newcommand{\AnnotationTok}[1]{\textcolor[rgb]{0.56,0.35,0.01}{\textbf{\textit{#1}}}}
\newcommand{\AttributeTok}[1]{\textcolor[rgb]{0.13,0.29,0.53}{#1}}
\newcommand{\BaseNTok}[1]{\textcolor[rgb]{0.00,0.00,0.81}{#1}}
\newcommand{\BuiltInTok}[1]{#1}
\newcommand{\CharTok}[1]{\textcolor[rgb]{0.31,0.60,0.02}{#1}}
\newcommand{\CommentTok}[1]{\textcolor[rgb]{0.56,0.35,0.01}{\textit{#1}}}
\newcommand{\CommentVarTok}[1]{\textcolor[rgb]{0.56,0.35,0.01}{\textbf{\textit{#1}}}}
\newcommand{\ConstantTok}[1]{\textcolor[rgb]{0.56,0.35,0.01}{#1}}
\newcommand{\ControlFlowTok}[1]{\textcolor[rgb]{0.13,0.29,0.53}{\textbf{#1}}}
\newcommand{\DataTypeTok}[1]{\textcolor[rgb]{0.13,0.29,0.53}{#1}}
\newcommand{\DecValTok}[1]{\textcolor[rgb]{0.00,0.00,0.81}{#1}}
\newcommand{\DocumentationTok}[1]{\textcolor[rgb]{0.56,0.35,0.01}{\textbf{\textit{#1}}}}
\newcommand{\ErrorTok}[1]{\textcolor[rgb]{0.64,0.00,0.00}{\textbf{#1}}}
\newcommand{\ExtensionTok}[1]{#1}
\newcommand{\FloatTok}[1]{\textcolor[rgb]{0.00,0.00,0.81}{#1}}
\newcommand{\FunctionTok}[1]{\textcolor[rgb]{0.13,0.29,0.53}{\textbf{#1}}}
\newcommand{\ImportTok}[1]{#1}
\newcommand{\InformationTok}[1]{\textcolor[rgb]{0.56,0.35,0.01}{\textbf{\textit{#1}}}}
\newcommand{\KeywordTok}[1]{\textcolor[rgb]{0.13,0.29,0.53}{\textbf{#1}}}
\newcommand{\NormalTok}[1]{#1}
\newcommand{\OperatorTok}[1]{\textcolor[rgb]{0.81,0.36,0.00}{\textbf{#1}}}
\newcommand{\OtherTok}[1]{\textcolor[rgb]{0.56,0.35,0.01}{#1}}
\newcommand{\PreprocessorTok}[1]{\textcolor[rgb]{0.56,0.35,0.01}{\textit{#1}}}
\newcommand{\RegionMarkerTok}[1]{#1}
\newcommand{\SpecialCharTok}[1]{\textcolor[rgb]{0.81,0.36,0.00}{\textbf{#1}}}
\newcommand{\SpecialStringTok}[1]{\textcolor[rgb]{0.31,0.60,0.02}{#1}}
\newcommand{\StringTok}[1]{\textcolor[rgb]{0.31,0.60,0.02}{#1}}
\newcommand{\VariableTok}[1]{\textcolor[rgb]{0.00,0.00,0.00}{#1}}
\newcommand{\VerbatimStringTok}[1]{\textcolor[rgb]{0.31,0.60,0.02}{#1}}
\newcommand{\WarningTok}[1]{\textcolor[rgb]{0.56,0.35,0.01}{\textbf{\textit{#1}}}}
\usepackage{graphicx}
\makeatletter
\def\maxwidth{\ifdim\Gin@nat@width>\linewidth\linewidth\else\Gin@nat@width\fi}
\def\maxheight{\ifdim\Gin@nat@height>\textheight\textheight\else\Gin@nat@height\fi}
\makeatother
% Scale images if necessary, so that they will not overflow the page
% margins by default, and it is still possible to overwrite the defaults
% using explicit options in \includegraphics[width, height, ...]{}
\setkeys{Gin}{width=\maxwidth,height=\maxheight,keepaspectratio}
% Set default figure placement to htbp
\makeatletter
\def\fps@figure{htbp}
\makeatother
\setlength{\emergencystretch}{3em} % prevent overfull lines
\providecommand{\tightlist}{%
  \setlength{\itemsep}{0pt}\setlength{\parskip}{0pt}}
\setcounter{secnumdepth}{-\maxdimen} % remove section numbering
\ifLuaTeX
\usepackage[bidi=basic]{babel}
\else
\usepackage[bidi=default]{babel}
\fi
\babelprovide[main,import]{russian}
% get rid of language-specific shorthands (see #6817):
\let\LanguageShortHands\languageshorthands
\def\languageshorthands#1{}
\ifLuaTeX
  \usepackage{selnolig}  % disable illegal ligatures
\fi
\IfFileExists{bookmark.sty}{\usepackage{bookmark}}{\usepackage{hyperref}}
\IfFileExists{xurl.sty}{\usepackage{xurl}}{} % add URL line breaks if available
\urlstyle{same}
\hypersetup{
  pdftitle={Тесты единичного корня. Расширенный тест Дики-Фуллера},
  pdfauthor={Достовалов М.Ю.},
  pdflang={ru-russian},
  hidelinks,
  pdfcreator={LaTeX via pandoc}}

\title{Тесты единичного корня. Расширенный тест Дики-Фуллера}
\author{Достовалов М.Ю.}
\date{07.03.2024}

\begin{document}
\maketitle

Установим необходимые пакеты и подгрузим библиотеки

\begin{Shaded}
\begin{Highlighting}[]
\CommentTok{\# install.packages(\textquotesingle{}urca\textquotesingle{})}
\CommentTok{\# install.packages(\textquotesingle{}aTSA\textquotesingle{})}

\FunctionTok{library}\NormalTok{(aTSA)}
\FunctionTok{library}\NormalTok{(urca)}
\FunctionTok{library}\NormalTok{(haven)}
\FunctionTok{library}\NormalTok{(stats)}
\FunctionTok{library}\NormalTok{(tseries)}
\end{Highlighting}
\end{Shaded}

Задание 1. Даны случайные процессы у1, у2, у3, y4, у21. Файл: DF.dta.
Cформулируйте и проверьте гипотезу о наличие единичного корня. Запишите
тестируемую регрессию в критерии Дики-Фуллера для y1 .

Загрузим данные из файла

\begin{Shaded}
\begin{Highlighting}[]
\CommentTok{\#file.choose()}
\NormalTok{DF}\OtherTok{=}\FunctionTok{read\_dta}\NormalTok{(}\StringTok{"DF.dta"}\NormalTok{) }\CommentTok{\# укажите свой путь, где лежит файл}
\end{Highlighting}
\end{Shaded}

\begin{enumerate}
\def\labelenumi{\arabic{enumi}.}
\tightlist
\item
  Построим графики исходного временного ряда для процесса y1
\end{enumerate}

\begin{Shaded}
\begin{Highlighting}[]
\FunctionTok{plot.ts}\NormalTok{(DF}\SpecialCharTok{$}\NormalTok{y1, }\AttributeTok{xlab=}\StringTok{"t"}\NormalTok{)}
\end{Highlighting}
\end{Shaded}

\includegraphics{Seminar_8_DF-test-with-R_files/figure-latex/unnamed-chunk-3-1.pdf}

Проведем тест Дики-Фуллера (с константой) для у1

\begin{Shaded}
\begin{Highlighting}[]
\FunctionTok{summary}\NormalTok{(}\FunctionTok{ur.df}\NormalTok{(DF}\SpecialCharTok{$}\NormalTok{y1, }\AttributeTok{type =} \FunctionTok{c}\NormalTok{(}\StringTok{"drift"}\NormalTok{), }\AttributeTok{lags =} \DecValTok{0}\NormalTok{))}
\end{Highlighting}
\end{Shaded}

\begin{verbatim}
## 
## ############################################### 
## # Augmented Dickey-Fuller Test Unit Root Test # 
## ############################################### 
## 
## Test regression drift 
## 
## 
## Call:
## lm(formula = z.diff ~ z.lag.1 + 1)
## 
## Residuals:
##     Min      1Q  Median      3Q     Max 
## -2.0004 -0.7543 -0.1044  0.5796  3.2531 
## 
## Coefficients:
##             Estimate Std. Error t value Pr(>|t|)    
## (Intercept) -0.09254    0.10837  -0.854    0.395    
## z.lag.1     -0.93032    0.10348  -8.990 2.03e-14 ***
## ---
## Signif. codes:  0 '***' 0.001 '**' 0.01 '*' 0.05 '.' 0.1 ' ' 1
## 
## Residual standard error: 1.07 on 97 degrees of freedom
## Multiple R-squared:  0.4545, Adjusted R-squared:  0.4489 
## F-statistic: 80.83 on 1 and 97 DF,  p-value: 2.027e-14
## 
## 
## Value of test-statistic is: -8.9905 40.4569 
## 
## Critical values for test statistics: 
##       1pct  5pct 10pct
## tau2 -3.51 -2.89 -2.58
## phi1  6.70  4.71  3.86
\end{verbatim}

Проведем тест Дики-Фуллера (с константой и с трендом) для у1

\begin{Shaded}
\begin{Highlighting}[]
\FunctionTok{summary}\NormalTok{(}\FunctionTok{ur.df}\NormalTok{(DF}\SpecialCharTok{$}\NormalTok{y1, }\AttributeTok{type =} \FunctionTok{c}\NormalTok{(}\StringTok{"trend"}\NormalTok{), }\AttributeTok{lags =} \DecValTok{0}\NormalTok{))}
\end{Highlighting}
\end{Shaded}

\begin{verbatim}
## 
## ############################################### 
## # Augmented Dickey-Fuller Test Unit Root Test # 
## ############################################### 
## 
## Test regression trend 
## 
## 
## Call:
## lm(formula = z.diff ~ z.lag.1 + 1 + tt)
## 
## Residuals:
##     Min      1Q  Median      3Q     Max 
## -1.9884 -0.6923 -0.1177  0.6070  3.1228 
## 
## Coefficients:
##              Estimate Std. Error t value Pr(>|t|)    
## (Intercept) -0.278886   0.218630  -1.276    0.205    
## z.lag.1     -0.938652   0.103846  -9.039 1.73e-14 ***
## tt           0.003705   0.003775   0.981    0.329    
## ---
## Signif. codes:  0 '***' 0.001 '**' 0.01 '*' 0.05 '.' 0.1 ' ' 1
## 
## Residual standard error: 1.07 on 96 degrees of freedom
## Multiple R-squared:  0.4599, Adjusted R-squared:  0.4487 
## F-statistic: 40.88 on 2 and 96 DF,  p-value: 1.435e-13
## 
## 
## Value of test-statistic is: -9.0389 27.2821 40.8806 
## 
## Critical values for test statistics: 
##       1pct  5pct 10pct
## tau3 -4.04 -3.45 -3.15
## phi2  6.50  4.88  4.16
## phi3  8.73  6.49  5.47
\end{verbatim}

Проведем расширенный тест Дики-Фуллера (с 1 дополнительным лагом) для у1

\begin{Shaded}
\begin{Highlighting}[]
\FunctionTok{summary}\NormalTok{(}\FunctionTok{ur.df}\NormalTok{(DF}\SpecialCharTok{$}\NormalTok{y1, }\AttributeTok{type =} \FunctionTok{c}\NormalTok{(}\StringTok{"drift"}\NormalTok{), }\AttributeTok{lags =} \DecValTok{1}\NormalTok{))}
\end{Highlighting}
\end{Shaded}

\begin{verbatim}
## 
## ############################################### 
## # Augmented Dickey-Fuller Test Unit Root Test # 
## ############################################### 
## 
## Test regression drift 
## 
## 
## Call:
## lm(formula = z.diff ~ z.lag.1 + 1 + z.diff.lag)
## 
## Residuals:
##     Min      1Q  Median      3Q     Max 
## -2.0503 -0.7865 -0.1247  0.5915  3.2679 
## 
## Coefficients:
##             Estimate Std. Error t value Pr(>|t|)    
## (Intercept) -0.08384    0.11051  -0.759    0.450    
## z.lag.1     -0.88443    0.14258  -6.203 1.44e-08 ***
## z.diff.lag  -0.05180    0.10473  -0.495    0.622    
## ---
## Signif. codes:  0 '***' 0.001 '**' 0.01 '*' 0.05 '.' 0.1 ' ' 1
## 
## Residual standard error: 1.079 on 95 degrees of freedom
## Multiple R-squared:  0.4557, Adjusted R-squared:  0.4442 
## F-statistic: 39.76 on 2 and 95 DF,  p-value: 2.841e-13
## 
## 
## Value of test-statistic is: -6.2028 19.2736 
## 
## Critical values for test statistics: 
##       1pct  5pct 10pct
## tau2 -3.51 -2.89 -2.58
## phi1  6.70  4.71  3.86
\end{verbatim}

Проведем расширенный тест Дики-Фуллера (с 2 дополнительными лагами) для
у1

\begin{Shaded}
\begin{Highlighting}[]
\FunctionTok{summary}\NormalTok{(}\FunctionTok{ur.df}\NormalTok{(DF}\SpecialCharTok{$}\NormalTok{y1, }\AttributeTok{type =} \FunctionTok{c}\NormalTok{(}\StringTok{"drift"}\NormalTok{), }\AttributeTok{lags =} \DecValTok{2}\NormalTok{))}
\end{Highlighting}
\end{Shaded}

\begin{verbatim}
## 
## ############################################### 
## # Augmented Dickey-Fuller Test Unit Root Test # 
## ############################################### 
## 
## Test regression drift 
## 
## 
## Call:
## lm(formula = z.diff ~ z.lag.1 + 1 + z.diff.lag)
## 
## Residuals:
##     Min      1Q  Median      3Q     Max 
## -2.1837 -0.7007 -0.1126  0.5773  3.0469 
## 
## Coefficients:
##             Estimate Std. Error t value Pr(>|t|)    
## (Intercept) -0.10777    0.11171  -0.965    0.337    
## z.lag.1     -1.02493    0.17205  -5.957 4.54e-08 ***
## z.diff.lag1  0.09168    0.14328   0.640    0.524    
## z.diff.lag2  0.15794    0.10602   1.490    0.140    
## ---
## Signif. codes:  0 '***' 0.001 '**' 0.01 '*' 0.05 '.' 0.1 ' ' 1
## 
## Residual standard error: 1.077 on 93 degrees of freedom
## Multiple R-squared:  0.468,  Adjusted R-squared:  0.4509 
## F-statistic: 27.27 on 3 and 93 DF,  p-value: 9.623e-13
## 
## 
## Value of test-statistic is: -5.9573 17.7741 
## 
## Critical values for test statistics: 
##       1pct  5pct 10pct
## tau2 -3.51 -2.89 -2.58
## phi1  6.70  4.71  3.86
\end{verbatim}

Проведем тест Дики-Фуллера (с оптимальным количеством лагов) для у1

\begin{Shaded}
\begin{Highlighting}[]
\FunctionTok{summary}\NormalTok{(}\FunctionTok{ur.df}\NormalTok{(DF}\SpecialCharTok{$}\NormalTok{y1, }\AttributeTok{type =} \FunctionTok{c}\NormalTok{(}\StringTok{"drift"}\NormalTok{)))}
\end{Highlighting}
\end{Shaded}

\begin{verbatim}
## 
## ############################################### 
## # Augmented Dickey-Fuller Test Unit Root Test # 
## ############################################### 
## 
## Test regression drift 
## 
## 
## Call:
## lm(formula = z.diff ~ z.lag.1 + 1 + z.diff.lag)
## 
## Residuals:
##     Min      1Q  Median      3Q     Max 
## -2.0503 -0.7865 -0.1247  0.5915  3.2679 
## 
## Coefficients:
##             Estimate Std. Error t value Pr(>|t|)    
## (Intercept) -0.08384    0.11051  -0.759    0.450    
## z.lag.1     -0.88443    0.14258  -6.203 1.44e-08 ***
## z.diff.lag  -0.05180    0.10473  -0.495    0.622    
## ---
## Signif. codes:  0 '***' 0.001 '**' 0.01 '*' 0.05 '.' 0.1 ' ' 1
## 
## Residual standard error: 1.079 on 95 degrees of freedom
## Multiple R-squared:  0.4557, Adjusted R-squared:  0.4442 
## F-statistic: 39.76 on 2 and 95 DF,  p-value: 2.841e-13
## 
## 
## Value of test-statistic is: -6.2028 19.2736 
## 
## Critical values for test statistics: 
##       1pct  5pct 10pct
## tau2 -3.51 -2.89 -2.58
## phi1  6.70  4.71  3.86
\end{verbatim}

Задание 2. Разностно-стационарные ряды. Исследуйте y2 и y21. Используйте
тест Дики-Фуллера для первой разности изучаемых процессов (в случае
необходимости). Сделайте вывод о порядке интегрируемости процессов
(после какой разности процесс стал стационарным).

Построим графики исходного временного ряда для процесса y2

\begin{Shaded}
\begin{Highlighting}[]
\FunctionTok{plot.ts}\NormalTok{(DF}\SpecialCharTok{$}\NormalTok{y2, }\AttributeTok{xlab=}\StringTok{"t"}\NormalTok{)}
\end{Highlighting}
\end{Shaded}

\includegraphics{Seminar_8_DF-test-with-R_files/figure-latex/unnamed-chunk-9-1.pdf}

Проведем тест Дики-Фуллера (с константой и с трендом) для у2

\begin{Shaded}
\begin{Highlighting}[]
\FunctionTok{summary}\NormalTok{(}\FunctionTok{ur.df}\NormalTok{(DF}\SpecialCharTok{$}\NormalTok{y2, }\AttributeTok{type =} \FunctionTok{c}\NormalTok{(}\StringTok{"trend"}\NormalTok{)))}
\end{Highlighting}
\end{Shaded}

\begin{verbatim}
## 
## ############################################### 
## # Augmented Dickey-Fuller Test Unit Root Test # 
## ############################################### 
## 
## Test regression trend 
## 
## 
## Call:
## lm(formula = z.diff ~ z.lag.1 + 1 + tt + z.diff.lag)
## 
## Residuals:
##      Min       1Q   Median       3Q      Max 
## -1.91983 -0.69289 -0.09198  0.66596  3.05289 
## 
## Coefficients:
##              Estimate Std. Error t value Pr(>|t|)
## (Intercept) -0.208187   0.241721  -0.861    0.391
## z.lag.1     -0.028729   0.038461  -0.747    0.457
## tt          -0.002847   0.009517  -0.299    0.765
## z.diff.lag   0.080654   0.108164   0.746    0.458
## 
## Residual standard error: 1.078 on 94 degrees of freedom
## Multiple R-squared:  0.01953,    Adjusted R-squared:  -0.01176 
## F-statistic: 0.6242 on 3 and 94 DF,  p-value: 0.6011
## 
## 
## Value of test-statistic is: -0.7469 0.7096 0.7267 
## 
## Critical values for test statistics: 
##       1pct  5pct 10pct
## tau3 -4.04 -3.45 -3.15
## phi2  6.50  4.88  4.16
## phi3  8.73  6.49  5.47
\end{verbatim}

Проведем тест Дики-Фуллера (с константой и без тренда) для у2

\begin{Shaded}
\begin{Highlighting}[]
\FunctionTok{summary}\NormalTok{(}\FunctionTok{ur.df}\NormalTok{(DF}\SpecialCharTok{$}\NormalTok{y2, }\AttributeTok{type =} \FunctionTok{c}\NormalTok{(}\StringTok{"drift"}\NormalTok{)))}
\end{Highlighting}
\end{Shaded}

\begin{verbatim}
## 
## ############################################### 
## # Augmented Dickey-Fuller Test Unit Root Test # 
## ############################################### 
## 
## Test regression drift 
## 
## 
## Call:
## lm(formula = z.diff ~ z.lag.1 + 1 + z.diff.lag)
## 
## Residuals:
##     Min      1Q  Median      3Q     Max 
## -1.9243 -0.6817 -0.1260  0.6465  3.0627 
## 
## Coefficients:
##             Estimate Std. Error t value Pr(>|t|)
## (Intercept) -0.25652    0.17893  -1.434    0.155
## z.lag.1     -0.01821    0.01552  -1.174    0.244
## z.diff.lag   0.07250    0.10417   0.696    0.488
## 
## Residual standard error: 1.073 on 95 degrees of freedom
## Multiple R-squared:  0.0186, Adjusted R-squared:  -0.002063 
## F-statistic: 0.9002 on 2 and 95 DF,  p-value: 0.4099
## 
## 
## Value of test-statistic is: -1.1735 1.0295 
## 
## Critical values for test statistics: 
##       1pct  5pct 10pct
## tau2 -3.51 -2.89 -2.58
## phi1  6.70  4.71  3.86
\end{verbatim}

Проведем тест Дики-Фуллера (без константы и тренда) для у2

\begin{Shaded}
\begin{Highlighting}[]
\FunctionTok{summary}\NormalTok{(}\FunctionTok{ur.df}\NormalTok{(DF}\SpecialCharTok{$}\NormalTok{y2, }\AttributeTok{type =} \FunctionTok{c}\NormalTok{(}\StringTok{"none"}\NormalTok{)))}
\end{Highlighting}
\end{Shaded}

\begin{verbatim}
## 
## ############################################### 
## # Augmented Dickey-Fuller Test Unit Root Test # 
## ############################################### 
## 
## Test regression none 
## 
## 
## Call:
## lm(formula = z.diff ~ z.lag.1 - 1 + z.diff.lag)
## 
## Residuals:
##     Min      1Q  Median      3Q     Max 
## -2.1087 -0.8478 -0.1844  0.5008  3.1353 
## 
## Coefficients:
##              Estimate Std. Error t value Pr(>|t|)
## z.lag.1    -0.0005812  0.0095163  -0.061    0.951
## z.diff.lag  0.0787237  0.1046476   0.752    0.454
## 
## Residual standard error: 1.079 on 96 degrees of freedom
## Multiple R-squared:  0.005869,   Adjusted R-squared:  -0.01484 
## F-statistic: 0.2834 on 2 and 96 DF,  p-value: 0.7539
## 
## 
## Value of test-statistic is: -0.0611 
## 
## Critical values for test statistics: 
##      1pct  5pct 10pct
## tau1 -2.6 -1.95 -1.61
\end{verbatim}

Проведем тест Дики-Фуллера (с константой и трендом) для первой разности
у2

\begin{Shaded}
\begin{Highlighting}[]
\NormalTok{d\_y2 }\OtherTok{=} \FunctionTok{diff}\NormalTok{(DF}\SpecialCharTok{$}\NormalTok{y2)}
\FunctionTok{summary}\NormalTok{(}\FunctionTok{ur.df}\NormalTok{(d\_y2, }\AttributeTok{type =} \FunctionTok{c}\NormalTok{(}\StringTok{"trend"}\NormalTok{)))}
\end{Highlighting}
\end{Shaded}

\begin{verbatim}
## 
## ############################################### 
## # Augmented Dickey-Fuller Test Unit Root Test # 
## ############################################### 
## 
## Test regression trend 
## 
## 
## Call:
## lm(formula = z.diff ~ z.lag.1 + 1 + tt + z.diff.lag)
## 
## Residuals:
##     Min      1Q  Median      3Q     Max 
## -2.0114 -0.7226 -0.1792  0.5872  3.1354 
## 
## Coefficients:
##              Estimate Std. Error t value Pr(>|t|)    
## (Intercept) -0.277115   0.228409  -1.213    0.228    
## z.lag.1     -0.895190   0.144404  -6.199 1.54e-08 ***
## tt           0.003784   0.003952   0.957    0.341    
## z.diff.lag  -0.046562   0.105865  -0.440    0.661    
## ---
## Signif. codes:  0 '***' 0.001 '**' 0.01 '*' 0.05 '.' 0.1 ' ' 1
## 
## Residual standard error: 1.085 on 93 degrees of freedom
## Multiple R-squared:  0.4607, Adjusted R-squared:  0.4433 
## F-statistic: 26.48 on 3 and 93 DF,  p-value: 1.812e-12
## 
## 
## Value of test-statistic is: -6.1992 12.8663 19.2804 
## 
## Critical values for test statistics: 
##       1pct  5pct 10pct
## tau3 -4.04 -3.45 -3.15
## phi2  6.50  4.88  4.16
## phi3  8.73  6.49  5.47
\end{verbatim}

Вывод: После взятия первой разности процес у2 стал стационарным. Значит
это разностно-стационарный процесс с порядком интегрируемости I(1)

Построим графики исходного временного ряда для процесса y21

\begin{Shaded}
\begin{Highlighting}[]
\FunctionTok{plot.ts}\NormalTok{(DF}\SpecialCharTok{$}\NormalTok{y21, }\AttributeTok{xlab=}\StringTok{"t"}\NormalTok{)}
\end{Highlighting}
\end{Shaded}

\includegraphics{Seminar_8_DF-test-with-R_files/figure-latex/unnamed-chunk-14-1.pdf}

Проведем тест Дики-Фуллера (с константой и с трендом) для у21

\begin{Shaded}
\begin{Highlighting}[]
\FunctionTok{summary}\NormalTok{(}\FunctionTok{ur.df}\NormalTok{(DF}\SpecialCharTok{$}\NormalTok{y21, }\AttributeTok{type =} \FunctionTok{c}\NormalTok{(}\StringTok{"trend"}\NormalTok{)))}
\end{Highlighting}
\end{Shaded}

\begin{verbatim}
## 
## ############################################### 
## # Augmented Dickey-Fuller Test Unit Root Test # 
## ############################################### 
## 
## Test regression trend 
## 
## 
## Call:
## lm(formula = z.diff ~ z.lag.1 + 1 + tt + z.diff.lag)
## 
## Residuals:
##      Min       1Q   Median       3Q      Max 
## -2.43502 -0.51917  0.04679  0.62525  2.27102 
## 
## Coefficients:
##              Estimate Std. Error t value Pr(>|t|)    
## (Intercept)  0.207570   0.258902   0.802  0.42473    
## z.lag.1     -0.005736   0.002043  -2.807  0.00608 ** 
## tt           0.017583   0.008077   2.177  0.03199 *  
## z.diff.lag   0.982751   0.031869  30.837  < 2e-16 ***
## ---
## Signif. codes:  0 '***' 0.001 '**' 0.01 '*' 0.05 '.' 0.1 ' ' 1
## 
## Residual standard error: 0.8853 on 94 degrees of freedom
## Multiple R-squared:  0.9396, Adjusted R-squared:  0.9377 
## F-statistic: 487.6 on 3 and 94 DF,  p-value: < 2.2e-16
## 
## 
## Value of test-statistic is: -2.8072 2.965 4.147 
## 
## Critical values for test statistics: 
##       1pct  5pct 10pct
## tau3 -4.04 -3.45 -3.15
## phi2  6.50  4.88  4.16
## phi3  8.73  6.49  5.47
\end{verbatim}

Проведем тест Дики-Фуллера (с константой и без тренда) для у21

\begin{Shaded}
\begin{Highlighting}[]
\FunctionTok{summary}\NormalTok{(}\FunctionTok{ur.df}\NormalTok{(DF}\SpecialCharTok{$}\NormalTok{y21, }\AttributeTok{type =} \FunctionTok{c}\NormalTok{(}\StringTok{"drift"}\NormalTok{)))}
\end{Highlighting}
\end{Shaded}

\begin{verbatim}
## 
## ############################################### 
## # Augmented Dickey-Fuller Test Unit Root Test # 
## ############################################### 
## 
## Test regression drift 
## 
## 
## Call:
## lm(formula = z.diff ~ z.lag.1 + 1 + z.diff.lag)
## 
## Residuals:
##      Min       1Q   Median       3Q      Max 
## -2.45757 -0.54107  0.07336  0.64461  2.05510 
## 
## Coefficients:
##               Estimate Std. Error t value Pr(>|t|)    
## (Intercept)  0.4661693  0.2345236   1.988   0.0497 *  
## z.lag.1     -0.0018082  0.0009777  -1.849   0.0675 .  
## z.diff.lag   0.9471763  0.0278938  33.957   <2e-16 ***
## ---
## Signif. codes:  0 '***' 0.001 '**' 0.01 '*' 0.05 '.' 0.1 ' ' 1
## 
## Residual standard error: 0.9025 on 95 degrees of freedom
## Multiple R-squared:  0.9366, Adjusted R-squared:  0.9352 
## F-statistic: 701.4 on 2 and 95 DF,  p-value: < 2.2e-16
## 
## 
## Value of test-statistic is: -1.8495 1.9994 
## 
## Critical values for test statistics: 
##       1pct  5pct 10pct
## tau2 -3.51 -2.89 -2.58
## phi1  6.70  4.71  3.86
\end{verbatim}

Проведем тест Дики-Фуллера (без константы и тренда) для у21

\begin{Shaded}
\begin{Highlighting}[]
\FunctionTok{summary}\NormalTok{(}\FunctionTok{ur.df}\NormalTok{(DF}\SpecialCharTok{$}\NormalTok{y21, }\AttributeTok{type =} \FunctionTok{c}\NormalTok{(}\StringTok{"none"}\NormalTok{)))}
\end{Highlighting}
\end{Shaded}

\begin{verbatim}
## 
## ############################################### 
## # Augmented Dickey-Fuller Test Unit Root Test # 
## ############################################### 
## 
## Test regression none 
## 
## 
## Call:
## lm(formula = z.diff ~ z.lag.1 - 1 + z.diff.lag)
## 
## Residuals:
##      Min       1Q   Median       3Q      Max 
## -2.52097 -0.50711  0.07007  0.73891  2.01159 
## 
## Coefficients:
##              Estimate Std. Error t value Pr(>|t|)    
## z.lag.1    -0.0001024  0.0004756  -0.215     0.83    
## z.diff.lag  0.9798935  0.0228634  42.859   <2e-16 ***
## ---
## Signif. codes:  0 '***' 0.001 '**' 0.01 '*' 0.05 '.' 0.1 ' ' 1
## 
## Residual standard error: 0.9163 on 96 degrees of freedom
## Multiple R-squared:  0.9565, Adjusted R-squared:  0.9556 
## F-statistic:  1056 on 2 and 96 DF,  p-value: < 2.2e-16
## 
## 
## Value of test-statistic is: -0.2153 
## 
## Critical values for test statistics: 
##      1pct  5pct 10pct
## tau1 -2.6 -1.95 -1.61
\end{verbatim}

Проведем тест Дики-Фуллера (с константой и трендом) для первой разности
у21

\begin{Shaded}
\begin{Highlighting}[]
\NormalTok{d\_y21 }\OtherTok{=} \FunctionTok{diff}\NormalTok{(DF}\SpecialCharTok{$}\NormalTok{y21)}
\FunctionTok{summary}\NormalTok{(}\FunctionTok{ur.df}\NormalTok{(d\_y21, }\AttributeTok{type =} \FunctionTok{c}\NormalTok{(}\StringTok{"trend"}\NormalTok{)))}
\end{Highlighting}
\end{Shaded}

\begin{verbatim}
## 
## ############################################### 
## # Augmented Dickey-Fuller Test Unit Root Test # 
## ############################################### 
## 
## Test regression trend 
## 
## 
## Call:
## lm(formula = z.diff ~ z.lag.1 + 1 + tt + z.diff.lag)
## 
## Residuals:
##      Min       1Q   Median       3Q      Max 
## -2.57348 -0.54990  0.05764  0.73923  1.84196 
## 
## Coefficients:
##              Estimate Std. Error t value Pr(>|t|)
## (Intercept)  0.295471   0.273926   1.079    0.284
## z.lag.1     -0.050028   0.032212  -1.553    0.124
## tt          -0.003246   0.004046  -0.802    0.424
## z.diff.lag   0.075795   0.103728   0.731    0.467
## 
## Residual standard error: 0.9218 on 93 degrees of freedom
## Multiple R-squared:  0.02782,    Adjusted R-squared:  -0.003537 
## F-statistic: 0.8872 on 3 and 93 DF,  p-value: 0.4508
## 
## 
## Value of test-statistic is: -1.5531 0.808 1.2099 
## 
## Critical values for test statistics: 
##       1pct  5pct 10pct
## tau3 -4.04 -3.45 -3.15
## phi2  6.50  4.88  4.16
## phi3  8.73  6.49  5.47
\end{verbatim}

Проведем тест Дики-Фуллера (с константой и трендом) для второй разности
у21

\begin{Shaded}
\begin{Highlighting}[]
\NormalTok{d2\_y21 }\OtherTok{=} \FunctionTok{diff}\NormalTok{(d\_y21)}
\FunctionTok{summary}\NormalTok{(}\FunctionTok{ur.df}\NormalTok{(d2\_y21, }\AttributeTok{type =} \FunctionTok{c}\NormalTok{(}\StringTok{"trend"}\NormalTok{)))}
\end{Highlighting}
\end{Shaded}

\begin{verbatim}
## 
## ############################################### 
## # Augmented Dickey-Fuller Test Unit Root Test # 
## ############################################### 
## 
## Test regression trend 
## 
## 
## Call:
## lm(formula = z.diff ~ z.lag.1 + 1 + tt + z.diff.lag)
## 
## Residuals:
##      Min       1Q   Median       3Q      Max 
## -2.61954 -0.48001 -0.01269  0.67643  1.96672 
## 
## Coefficients:
##               Estimate Std. Error t value Pr(>|t|)    
## (Intercept)  0.0297342  0.1950024   0.152    0.879    
## z.lag.1     -0.9200948  0.1438528  -6.396 6.54e-09 ***
## tt          -0.0002792  0.0034376  -0.081    0.935    
## z.diff.lag  -0.0367042  0.1041295  -0.352    0.725    
## ---
## Signif. codes:  0 '***' 0.001 '**' 0.01 '*' 0.05 '.' 0.1 ' ' 1
## 
## Residual standard error: 0.9333 on 92 degrees of freedom
## Multiple R-squared:  0.4794, Adjusted R-squared:  0.4624 
## F-statistic: 28.24 on 3 and 92 DF,  p-value: 4.936e-13
## 
## 
## Value of test-statistic is: -6.3961 13.6567 20.4551 
## 
## Critical values for test statistics: 
##       1pct  5pct 10pct
## tau3 -4.04 -3.45 -3.15
## phi2  6.50  4.88  4.16
## phi3  8.73  6.49  5.47
\end{verbatim}

Вывод: После взятия второй разности процес у2 стал стационарным. Значит
это разностно-стационарный процесс с порядком интегрируемости I(2)

Задание 3. Тренд-стационарный ряд. Исследуйте y3. Предположив наличие в
процессах детерминированного тренда, проведите тест Дики-Фуллера.
Сделайте вывод.

Построим графики исходного временного ряда для процесса y3

\begin{Shaded}
\begin{Highlighting}[]
\FunctionTok{plot.ts}\NormalTok{(DF}\SpecialCharTok{$}\NormalTok{y3, }\AttributeTok{xlab=}\StringTok{"t"}\NormalTok{)}
\end{Highlighting}
\end{Shaded}

\includegraphics{Seminar_8_DF-test-with-R_files/figure-latex/unnamed-chunk-20-1.pdf}

Проведем тест Дики-Фуллера (с константой и без тренда) для у3

\begin{Shaded}
\begin{Highlighting}[]
\FunctionTok{summary}\NormalTok{(}\FunctionTok{ur.df}\NormalTok{(DF}\SpecialCharTok{$}\NormalTok{y3, }\AttributeTok{type =} \FunctionTok{c}\NormalTok{(}\StringTok{"drift"}\NormalTok{)))}
\end{Highlighting}
\end{Shaded}

\begin{verbatim}
## 
## ############################################### 
## # Augmented Dickey-Fuller Test Unit Root Test # 
## ############################################### 
## 
## Test regression drift 
## 
## 
## Call:
## lm(formula = z.diff ~ z.lag.1 + 1 + z.diff.lag)
## 
## Residuals:
##     Min      1Q  Median      3Q     Max 
## -3.3178 -0.8977  0.1058  0.9004  2.4986 
## 
## Coefficients:
##              Estimate Std. Error t value Pr(>|t|)    
## (Intercept) -0.826496   0.263321  -3.139  0.00226 ** 
## z.lag.1     -0.004451   0.009163  -0.486  0.62828    
## z.diff.lag  -0.489941   0.091231  -5.370 5.57e-07 ***
## ---
## Signif. codes:  0 '***' 0.001 '**' 0.01 '*' 0.05 '.' 0.1 ' ' 1
## 
## Residual standard error: 1.277 on 95 degrees of freedom
## Multiple R-squared:  0.2371, Adjusted R-squared:  0.221 
## F-statistic: 14.76 on 2 and 95 DF,  p-value: 2.612e-06
## 
## 
## Value of test-statistic is: -0.4857 13.8863 
## 
## Critical values for test statistics: 
##       1pct  5pct 10pct
## tau2 -3.51 -2.89 -2.58
## phi1  6.70  4.71  3.86
\end{verbatim}

Проведем тест Дики-Фуллера (c константой и трендом) для у3

\begin{Shaded}
\begin{Highlighting}[]
\FunctionTok{summary}\NormalTok{(}\FunctionTok{ur.df}\NormalTok{(DF}\SpecialCharTok{$}\NormalTok{y3, }\AttributeTok{type =} \FunctionTok{c}\NormalTok{(}\StringTok{"trend"}\NormalTok{)))}
\end{Highlighting}
\end{Shaded}

\begin{verbatim}
## 
## ############################################### 
## # Augmented Dickey-Fuller Test Unit Root Test # 
## ############################################### 
## 
## Test regression trend 
## 
## 
## Call:
## lm(formula = z.diff ~ z.lag.1 + 1 + tt + z.diff.lag)
## 
## Residuals:
##     Min      1Q  Median      3Q     Max 
## -2.0159 -0.7115 -0.1722  0.5739  3.1416 
## 
## Coefficients:
##             Estimate Std. Error t value Pr(>|t|)    
## (Intercept) -0.33601    0.23612  -1.423    0.158    
## z.lag.1     -0.89838    0.14355  -6.259 1.15e-08 ***
## tt          -0.44567    0.07146  -6.237 1.27e-08 ***
## z.diff.lag  -0.04391    0.10519  -0.417    0.677    
## ---
## Signif. codes:  0 '***' 0.001 '**' 0.01 '*' 0.05 '.' 0.1 ' ' 1
## 
## Residual standard error: 1.08 on 94 degrees of freedom
## Multiple R-squared:  0.4604, Adjusted R-squared:  0.4432 
## F-statistic: 26.73 on 3 and 94 DF,  p-value: 1.37e-12
## 
## 
## Value of test-statistic is: -6.2585 25.9154 19.6126 
## 
## Critical values for test statistics: 
##       1pct  5pct 10pct
## tau3 -4.04 -3.45 -3.15
## phi2  6.50  4.88  4.16
## phi3  8.73  6.49  5.47
\end{verbatim}

Вывод: Так как после добавления в модель тренда процесс стал
стационарным, то это тренд-стационарный процесс.

Задание 4. Альтернативные тесты единичного корня. Проведите PP- и
KPSS-тесты для y1, сравните результаты.

Проведем тест Дики-Фуллера для у1

\begin{Shaded}
\begin{Highlighting}[]
\FunctionTok{adf.test}\NormalTok{(DF}\SpecialCharTok{$}\NormalTok{y1)}
\end{Highlighting}
\end{Shaded}

\begin{verbatim}
## Warning in adf.test(DF$y1): p-value smaller than printed p-value
\end{verbatim}

\begin{verbatim}
## 
##  Augmented Dickey-Fuller Test
## 
## data:  DF$y1
## Dickey-Fuller = -4.3072, Lag order = 4, p-value = 0.01
## alternative hypothesis: stationary
\end{verbatim}

\begin{Shaded}
\begin{Highlighting}[]
\CommentTok{\#stationary.test(DF$y1, method = "adf")}
\end{Highlighting}
\end{Shaded}

Проведем PP-тест Филлипса-Перрона для у1

\begin{Shaded}
\begin{Highlighting}[]
\FunctionTok{pp.test}\NormalTok{(DF}\SpecialCharTok{$}\NormalTok{y1)}
\end{Highlighting}
\end{Shaded}

\begin{verbatim}
## Warning in pp.test(DF$y1): p-value smaller than printed p-value
\end{verbatim}

\begin{verbatim}
## 
##  Phillips-Perron Unit Root Test
## 
## data:  DF$y1
## Dickey-Fuller Z(alpha) = -91.416, Truncation lag parameter = 3, p-value
## = 0.01
## alternative hypothesis: stationary
\end{verbatim}

\begin{Shaded}
\begin{Highlighting}[]
\FunctionTok{stationary.test}\NormalTok{(DF}\SpecialCharTok{$}\NormalTok{y1, }\AttributeTok{method =} \StringTok{"pp"}\NormalTok{)}
\end{Highlighting}
\end{Shaded}

\begin{verbatim}
## Phillips-Perron Unit Root Test 
## alternative: stationary 
##  
## Type 1: no drift no trend 
##  lag Z_rho p.value
##    3 -90.3    0.01
## ----- 
##  Type 2: with drift no trend 
##  lag Z_rho p.value
##    3 -90.9    0.01
## ----- 
##  Type 3: with drift and trend 
##  lag Z_rho p.value
##    3 -91.4    0.01
## --------------- 
## Note: p-value = 0.01 means p.value <= 0.01
\end{verbatim}

Проведем тест KPSS (Квятковского-Филлипса-Шмидта-Шина) для у1

\begin{Shaded}
\begin{Highlighting}[]
\FunctionTok{kpss.test}\NormalTok{(DF}\SpecialCharTok{$}\NormalTok{y1)}
\end{Highlighting}
\end{Shaded}

\begin{verbatim}
## Warning in kpss.test(DF$y1): p-value greater than printed p-value
\end{verbatim}

\begin{verbatim}
## 
##  KPSS Test for Level Stationarity
## 
## data:  DF$y1
## KPSS Level = 0.26643, Truncation lag parameter = 4, p-value = 0.1
\end{verbatim}

\begin{Shaded}
\begin{Highlighting}[]
\CommentTok{\#stationary.test(DF$y1, method = "kpss")}
\end{Highlighting}
\end{Shaded}

Задание 5. Тесты единичного корня с учетом структурного сдвига.
Проведите тесты единичного корня для у4, в предположении наличия
структурного сдвига. Сделайте выводы.

Построим графики исходного временного ряда для процесса y4

\begin{Shaded}
\begin{Highlighting}[]
\FunctionTok{plot.ts}\NormalTok{(DF}\SpecialCharTok{$}\NormalTok{y4, }\AttributeTok{xlab=}\StringTok{"t"}\NormalTok{)}
\end{Highlighting}
\end{Shaded}

\includegraphics{Seminar_8_DF-test-with-R_files/figure-latex/unnamed-chunk-26-1.pdf}

Проведем тест Эндрюса-Зивота для у4 с учетом структурного сдвига 1-го
типа (на константу)

\begin{Shaded}
\begin{Highlighting}[]
\FunctionTok{summary}\NormalTok{(}\FunctionTok{ur.za}\NormalTok{(DF}\SpecialCharTok{$}\NormalTok{y4, }\AttributeTok{model =} \FunctionTok{c}\NormalTok{(}\StringTok{"intercept"}\NormalTok{)))}
\end{Highlighting}
\end{Shaded}

\begin{verbatim}
## 
## ################################ 
## # Zivot-Andrews Unit Root Test # 
## ################################ 
## 
## 
## Call:
## lm(formula = testmat)
## 
## Residuals:
##      Min       1Q   Median       3Q      Max 
## -1.77807 -0.71987 -0.03273  0.54846  2.68554 
## 
## Coefficients:
##             Estimate Std. Error t value Pr(>|t|)    
## (Intercept) -0.04168    0.19747  -0.211    0.833    
## y.l1        -0.01270    0.06879  -0.185    0.854    
## trend        0.21409    0.01640  13.054   <2e-16 ***
## du           9.30655    0.66847  13.922   <2e-16 ***
## ---
## Signif. codes:  0 '***' 0.001 '**' 0.01 '*' 0.05 '.' 0.1 ' ' 1
## 
## Residual standard error: 0.9396 on 95 degrees of freedom
##   (1 пропущенное наблюдение удалено)
## Multiple R-squared:  0.9911, Adjusted R-squared:  0.9908 
## F-statistic:  3507 on 3 and 95 DF,  p-value: < 2.2e-16
## 
## 
## Teststatistic: -14.7224 
## Critical values: 0.01= -5.34 0.05= -4.8 0.1= -4.58 
## 
## Potential break point at position: 30
\end{verbatim}

Проведем тест Эндрюса-Зивота для у4 с учетом структурного сдвига 2-го
типа (на наклон)

\begin{Shaded}
\begin{Highlighting}[]
\FunctionTok{summary}\NormalTok{(}\FunctionTok{ur.za}\NormalTok{(DF}\SpecialCharTok{$}\NormalTok{y4, }\AttributeTok{model =} \FunctionTok{c}\NormalTok{(}\StringTok{"trend"}\NormalTok{)))}
\end{Highlighting}
\end{Shaded}

\begin{verbatim}
## 
## ################################ 
## # Zivot-Andrews Unit Root Test # 
## ################################ 
## 
## 
## Call:
## lm(formula = testmat)
## 
## Residuals:
##     Min      1Q  Median      3Q     Max 
## -3.9976 -0.7741  0.0401  0.7352  7.3375 
## 
## Coefficients:
##             Estimate Std. Error t value Pr(>|t|)    
## (Intercept) -1.13600    0.57375  -1.980 0.050600 .  
## y.l1         0.62055    0.07965   7.791 8.29e-12 ***
## trend        0.19202    0.04334   4.430 2.52e-05 ***
## dt          -0.11098    0.03190  -3.479 0.000762 ***
## ---
## Signif. codes:  0 '***' 0.001 '**' 0.01 '*' 0.05 '.' 0.1 ' ' 1
## 
## Residual standard error: 1.543 on 95 degrees of freedom
##   (1 пропущенное наблюдение удалено)
## Multiple R-squared:  0.9759, Adjusted R-squared:  0.9751 
## F-statistic:  1280 on 3 and 95 DF,  p-value: < 2.2e-16
## 
## 
## Teststatistic: -4.7637 
## Critical values: 0.01= -4.93 0.05= -4.42 0.1= -4.11 
## 
## Potential break point at position: 42
\end{verbatim}

Проведем тест Эндрюса-Зивота для у4 с учетом структурного сдвига 3-го
типа (на константу и наклон)

\begin{Shaded}
\begin{Highlighting}[]
\FunctionTok{summary}\NormalTok{(}\FunctionTok{ur.za}\NormalTok{(DF}\SpecialCharTok{$}\NormalTok{y4, }\AttributeTok{model =} \FunctionTok{c}\NormalTok{(}\StringTok{"both"}\NormalTok{)))}
\end{Highlighting}
\end{Shaded}

\begin{verbatim}
## 
## ################################ 
## # Zivot-Andrews Unit Root Test # 
## ################################ 
## 
## 
## Call:
## lm(formula = testmat)
## 
## Residuals:
##      Min       1Q   Median       3Q      Max 
## -1.77535 -0.73794 -0.06437  0.50821  2.55813 
## 
## Coefficients:
##             Estimate Std. Error t value Pr(>|t|)    
## (Intercept)  0.11509    0.37947   0.303    0.762    
## y.l1        -0.01288    0.06907  -0.187    0.852    
## trend        0.20433    0.02602   7.852 6.52e-12 ***
## du           9.42061    0.71126  13.245  < 2e-16 ***
## dt           0.01050    0.02167   0.485    0.629    
## ---
## Signif. codes:  0 '***' 0.001 '**' 0.01 '*' 0.05 '.' 0.1 ' ' 1
## 
## Residual standard error: 0.9434 on 94 degrees of freedom
##   (1 пропущенное наблюдение удалено)
## Multiple R-squared:  0.9911, Adjusted R-squared:  0.9907 
## F-statistic:  2609 on 4 and 94 DF,  p-value: < 2.2e-16
## 
## 
## Teststatistic: -14.6654 
## Critical values: 0.01= -5.57 0.05= -5.08 0.1= -4.82 
## 
## Potential break point at position: 30
\end{verbatim}

\end{document}
